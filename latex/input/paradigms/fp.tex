\subsection[Functional]{Functional programming}

\frame{\tableofcontents[currentsection, currentsubsection]}

\subsubsection{Definition}

\begin{frame}{Functional programming}
	% https://www.geeksforgeeks.org/functional-programming-paradigm/
	% https://en.wikipedia.org/wiki/Functional_programming#Pure_functions
	Functional programming
	\begin{itemize}
		\item expresses its computations in the style of \hhl{mathematical functions}
		\item emphasizes 
		\begin{itemize}
			\item \hhl{expressions} \srem{(\enquote{is} something: a series of identifiers, literals and operators that reduces to a value)} 
			\item[over\hspace{-0.7em}]
			\item \hhl{statements} \srem{(\enquote{does} something, e.g. stores value, etc.)}
		\end{itemize} 
		$\lra$ \hhl{declarative} nature
		\item Data is \hhl{immutable} \srem{(instead of changing properties, I need to create copies with the changed property)}
		\item Avoids \hhl{side effects} \srem{(expressions should not change or depend on any external state)}  
	\end{itemize}
	%
\end{frame}

\begin{frame}{Examples}
	Languages made for FP (picture book examples):
	\begin{itemize}
		\item \proglang{Lisp} and derivates: \proglang{Common Lisp}, \proglang{Clojure}, \dots
		\item \proglang{Haskell}
		\item \proglang{OCaml}
		\item \proglang{F\#}
		\item \proglang{Wolfram Language} (Mathematica etc.)
		\item \dots
	\end{itemize}
	%
	With emphasis on FP:
	\begin{itemize}
		\item \proglang{JavaScript}
		\item \proglang{R}
		\item \dots
	\end{itemize}
	%
	Not designed for, but offering strong support for FP:
	\begin{itemize}
		\item \proglang{C++} \srem{(from \proglang{C++11} on)}
		\item \proglang{Perl}
		\item \dots
	\end{itemize}
\end{frame}

%\begin{frame}[t]{Referential transparency}
%	\only<+->{
%		No side effects implies referential transparency:
%	
%		\medskip
%		\begin{center}
%		An expression is \hhl{referential transparent} if it can be replaced with its value without changing the outcome of the program.
%		\end{center}
%	}
%
%	Which of the following functions are referential transparent?
%%	\only<+->{
%%		\bigskip
%%		In particular for functions: \hhl{$x = y$ $\Lra$ $f(x) = f(y)$}.
%%	}
%	\only<+>{
%		\inputminted[]{python}{code/paradigms/fp/referential_transparency.py}
%	}
%\end{frame}

\begin{frame}[t]{Pure functions}
	% https://en.wikipedia.org/wiki/Pure_function
	\only<+->{
		A function is called pure if
		\begin{enumerate}
			\item Same arguments $\Lra$ same return value \srem{($x=y\ \Lra\ f(x)=f(y)$)}
			\item The evaluation has no side effects \srem{(no change in non-local variables, ...)}
		\end{enumerate}
	}
	
	\bigskip
	\only<2->{Which of the following functions are pure?}
	%
	\begin{columns}[t]
		\column{0.5\textwidth}
		\only<2->{
			\inputminted[lastline=15]{python}{code/paradigms/fp/referential_transparency.py}
		}
		\column{0.5\textwidth}
		\only<2->{
			\inputminted[firstline=18]{python}{code/paradigms/fp/referential_transparency.py}
		}
		\only<3->{
			\bigskip
			\stress{Answer}: \texttt{f1} is pure; \texttt{f1}, \texttt{f3}, \texttt{f5} violate rule 2; \texttt{f4} violates rule 1.
		}
	\end{columns}
	
\end{frame}

\begin{frame}[t]{Non strict evaluation}{}
	\begin{itemize}
		\item Some functional programming languages use \hldefn{non-strict evaluation}: The arguments of a function are \emph{only} evaluated once the function is called.
		
		\only<2->{
			\medskip
			Example: \mintinline{python}{print(sqrt(sin(a**2)))}
			
			\medskip
			In a \hhl{strict} language \srem{(e.g. \proglang{Python}, \proglang{C++})}, we evaluate inside out: 
			%
			\begin{equation*}
				a\lmt a^2\lmt \sin{a^2} \lmt \sqrt{\sin{a^2}}
			\end{equation*}
			%
			In a \hhl{non-strict} language, the evaluation of the inner part is \stress{deferred}, until it is actually needed. 
		}
		\item<3->
		\medskip
		But \proglang{Python} actually has something similar in the concept of \hhl{\texttt{generators}}:
		\inputminted[]{python}{code/paradigms/fp/generators.py}
	
		%
		\item<4-> This allows for \hhl{infinite data structures} \srem{(which can be more practical than it sounds)}
	\end{itemize}
\end{frame}

\begin{frame}[t]{Memoization}
	\begin{itemize}
		\item<+-> Non strict evaluation together with \hldefn{sharing} (avoid repeated evaluation of the same expression) is called \hldefn{lazy evaluation}
		%
		\item<+-> Generally, functional programming can get cheap performance boosts by very simple \hldefn{memoization}: Storing the results of expensive \stress{pure} function calls in a cache
		
		%\medskip
		\only<+->{\inputminted[]{python}{code/paradigms/fp/memoization.py}}
	\end{itemize}
\end{frame}

\begin{frame}{Higher order functions}
	A \hldefn{higher order function} does one of the following:
	\begin{itemize}
		\item returns a function
		\item takes a function as an argument
	\end{itemize}
	Opposite: \hldefn{first-order function}.
	
	\medskip 
	\srem{Mathematical examples (usually called \defn{operators} or \defn{functionals}): differential operator, integration, \dots}
	
	\medskip
	Higher level functions are the FP answer to template methods in OOP \srem{(\enquote{configuring} object behavior by overriding methods in subclasses)}.
	
	\medskip
	Classic example of a higher order function: \texttt{map} \srem{(applies function to all elements in list)}:
	\inputminted[lastline=8]{python}{code/paradigms/fp/higher_order.py}
\end{frame}

\begin{frame}{Higher order functions II}
	A function that also returns a function:
	
	\inputminted[]{python}{code/paradigms/fp/higher_order_2.py}
\end{frame}

\begin{frame}{Type systems}
	\hhl{Types:}
	\begin{itemize}
		\item In OOP, \hhl{type} and \hhl{class} are often used interchangeably
		%\footnote{though one might use \defn{type} for abstract interfaces, vs \defn{class} for implementations} 
		\srem{(e.g. \texttt{"abc"} is of type \defn{string} = is an instance of the \texttt{str} class)}
		\item In FP we talk about \defn{type}s
		\item Complex types can be built from built in types \\
		\srem{(e.g. \mintinline{python}{List[Tuple[str, int]]}, we can also use \texttt{struct}s)}
		\item In many languages, types of variables, arguments, etc. have to be declared \srem{(e.g. \mintinline{python}{def len(List[float]) -> int})}
		\item Real FP languages usually have very powerful \hhl{type systems}%; FP languages can be both 
	\end{itemize}
	
%	\medskip
%	\hhl{Subtype polymorphism:}
%	\begin{itemize}	
%		\item Types can have subtypes \srem{(e.g. \mintinline{python}{List[int]} as a subtype of \mintinline{python}{List[Number]]})}; any function for the type works on the subtype \srem{(e.g. \mintinline{python}{def sum(List[Number) -> int} now works on \mintinline{python}{List[Int]} as well as \mintinline{python}{List[float]})}
%		\item The type specifies which functions can be applied \srem{(this was done by defining \emph{methods} in FP)}; subtyping becomes similar to inheritance
%	\end{itemize}
\end{frame}

\begin{frame}{Type systems and polymorphism}
	In FP, the type system allows to bring back some OOP thinking but is more flexible. Usually you can do some of the following:
	
	\bigskip
	\hhl{Single/multiple dispatch:}
	\begin{itemize}
		\item Can \hhl{overload} function definitions \srem{(e.g. define \mintinline{python}{def print(i: int)} differently from \mintinline{python}{def print(string: str)})}
		\item The right function is resolved based on the type at compile- or runtime 
	\end{itemize}
	
	\medskip
	\hhl{Parametric polymorphism:}
	\begin{itemize}
		\item Parameterize types in function signatures \srem{(e.g. \mintinline{python}{def first(List[a]) -> a}; \texttt{a} represents an arbitrary type)}
	\end{itemize}

	\medskip
	\hhl{Type classes:}
	\begin{itemize}
		\item Define a \enquote{type} by what functions it has to support \srem{(e.g. define \texttt{Duck} as anything that allows me to call the \texttt{quack} function on it)}
		\item Similar to a class with only abstract methods or duck-typing \srem{(if it looks like a duck and quacks like a duck, it's a duck)}
	\end{itemize}
\end{frame}
%
%
\begin{frame}[t]{Looping in functional programming}
	\only<1->{Let's consider a function that calculates $\sum_{i=0}^N i^2$:}
	%
	\only<+->{\inputminted[]{python}{code/paradigms/fp/sum_loop.py}}
	
	
	\only<+->{
		\bigskip
		This is a function, but does not follow the FP paradigm:
		\begin{itemize}
			\item More statements \srem{(assignments, loops, \dots)} than expressions
			\item The for loop segment is not free of side effects \srem{(value of \texttt{result} changes)}
			\item Repeated reassignments of \texttt{result} are frowned upon \srem{(or impossible)}
		\end{itemize}
	}
	\only<+->{
	\bigskip
		How to change this? $\lra$ Use \hhl{recursion}
		%
		\inputminted[]{python}{code/paradigms/fp/sum_head_recursion.py}
	}
	%Tail recursions $\lra$ Compiler optimization
\end{frame}

\begin{frame}[t]{Looping in functional programming}
	\only<1->{
		The previous example is called a \hhl{head recursion} \srem{(recursion before computation)}; using a \hhl{tail recursion} \srem{(recursion after computation)} is preferable due to better compiler optimization:
	}
	%
	\only<+->{
		\inputminted[]{python}{code/paradigms/fp/sum_tail_recursion.py}
	}
	\only<+->{
		\bigskip
		Another FP way is to use the higher level functions \hhl{\texttt{map}} and \hhl{\texttt{reduce}} together with anonymous functions (\hhl{lambda}):
		\inputminted[]{python}{code/paradigms/fp/sum_map_reduce.py}
	}
\end{frame}

\subsubsection{Signature moves}

\subsubsection{Strenghts and Weaknesses}

\begin{frame}{Strenghts and Weaknesses}
\begin{block}{Strengths}
	\begin{itemize}
		\item Proving things \hhl{mathematically} \srem{(referential transparency, \dots)}
		\item \hhl{Testability} \srem{(no object initializations and complex dependencies, pure functions)}
		\item Easy \hhl{debugging} \srem{(no hidden states)}
		\item Can be very \hhl{short} and \hhl{concise} $\lra$ easy to verify
		
		\item Sophisticated logical abstractions \srem{(using high level functions)} $\lra$ modularity, \hhl{code reuse}
		\item Performance boost because of easy \hhl{parallelization}
	\end{itemize}
\end{block}
\end{frame}

\begin{frame}{Strengths and Weaknesses}
\begin{block}{Weaknesses}
	\begin{itemize}
		\item Structuring code in terms of objects can feel more intuitive if logic \srem{(methods)} are strongly tied to data 
		\item Imperative algorithms might be easier to read and feel more natural than declarative notation
		\item FP might have a steeper \hhl{learning curve} \srem{(e.g. recursions instead of loops, \dots)}
		\item \hhl{Performance issues} : EXPAND ME
%		\begin{itemize}
			%\item Declarative nature somewhat less efficient in CPU usage compared to imperative styles
			%\item due to \hhl{immutability}: 
%		\end{itemize}
		\item Still \hhl{small user base outside of academia}, but gaining traction
	\end{itemize}
\end{block}
\end{frame}
