\subsection{OOP vs FP}

\begin{frame}[t]{Object oriented vs functional programming}
	Some key aspects to keep in mind:
	\begin{itemize}[<+->]
		\item $\text{FP} \neq \text{OOP} - \text{classes}$
		\item FP is \stress{not} the opposite of OOP: Both paradigms take opposite stances in several aspects: declarative vs imperative, mutable vs immutable, \dots{} $\Longrightarrow$ Not everything can be classified into one of these categories
		\item Rather: Two different \hhl{ways to think} and to approach problems \srem{$\longrightarrow$ see caveats at the beginning}
	\end{itemize}
	
	\only<+->{
		\medskip	
		In a multi-paradigm language, you can use the best of both worlds!
		\begin{itemize}[<+->]
			\item \hhl{OOP} has its classical use cases where there is strong coupling between data and methods and the bookkeeping is in the focus (especially of "real-world" objects)
			\item \hhl{FP} instead focuses on algorithms and \emph{doing} things
			\item Some people advocate \hhl{\enquote{OOP in the large, FP in the small}} \srem{(using OOP as the high level interface, using FP techniques for implementing the logic)}
		\end{itemize}
	}

	\only<+->{
		For example:
		\begin{itemize}
			\item Many complicated class structures implementing \hhl{manipulations} can be made more flexible with a system of high level functions, anonymous functions etc. \srem{(\texttt{pandas.DataFrame.apply})}
		\end{itemize}
	}
\end{frame}
